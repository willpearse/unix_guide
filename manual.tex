\documentclass[12pt]{report}
\renewcommand\thesection{\arabic{section}}
\usepackage[parfill]{parskip}
\usepackage{subcaption}
\usepackage{graphicx}
\usepackage{listings}
\lstset{language=bash}

\begin{document}

\title{Getting things done with BASH and UNIX}
\author{Will Pearse}
\maketitle

\section{Preamble}
I am neither a computer scientist nor a system adminstrator. I'm
simply someone who finds it fun to do things with large chunks of
data, and isn't afraid of error messages. None of what is in this
guide is definitive, and none of it is guaranteed to be
correct. However, what follows is the result of (literally) years of
experience, frustration, and joy. It is my hope that by reading this,
and thinking about its contents, I can pass on the first, minimise the
second, and maximise the third. This guide is intended for those who
are interested in computers, but would prefer to get their other tasks
(in my case, biology) done over endlessly re-doing things to make them
more computationally `neat'.

The most important thing when working with a computer is to
\emph{relax}. Computers do not make mistakes, we make mistakes, and
every single thing that goes wrong during your time working with a
remote system will be either your, or the system administrator's,
fault. Computers don't have personalities, they don't hold grudges,
and, most importantly, \emph{they don't care if you make mistakes}. If
you don't care, and the computer doesn't care, then you will find
yourself transported to a zen-like state of bliss where mistakes and
errors simply flow over you. When you realise that you make mistakes,
and that there's nothing wrong with that, you'll find yourself much
happier.

This document has two very important sections: an introduction to BASH
(and UNIX), and a guide on how to log into a remote system (SSH). The
next two sections are useful but not as fundamental; dip into these at
your leisure. I do not repeat any information in this document; the
final section contains additional tips about programs you have already
started using.
\section{Getting things done (BASH)}
A UNIX-like computer is a set of programs that all balance, at times
precariously, on top of the \emph{kernel}. The kernel is the boss, and
when we want to run programs, modify its settings, insert a USB disk,
etc., we need a way to communicate with it. A \emph{shell} allows us
to do that, and BASH is the most commonly-used shell. There are
others, but my advice is to stick with BASH since it's available
everywhere. BASH is on (almost) all Linux and MacOS computers (open
`Terminal' and you will see it); MacOS is based on a modified form of
UNIX, and Linux is essentially a \emph{very} modified form of UNIX
itself. There is a long and tangled history of UNIX-like software; you
may hear terms like POSIX-compliant being used, and it is almost
always acceptable to think of those things as UNIX-like and have no
problems.
\subsection{Moving around}
You are, at all times, in a directory of some sort, in the same way as
you're always in a directory in Windows Explorer or Mac's Finder. To
see your \emph{present working directory}, type \texttt{pwd} and press
enter. You have just run the \texttt{pwd} command, and the directory
is spat out at you as an \emph{absolute path}: directories are
separated in the path by \texttt{/}, and the first \texttt{/}
represents the \emph{root} of your file system. It is the presence of
this root in the path that makes it absolute.

Unless you specify an absolute path, BASH assumes you're referring to
something relative to your current position. Let's make a new
directory, and then move into it, using relative paths. \emph{Make} a
\emph{directory} (\texttt{mkdir my.new.directory}), then \emph{change
  directory} into it (\texttt{cd my.new.directory}). Verify your new
present working directory. You can \emph{list} the things in your
directory using \texttt{ls}; running that command now will do nothing,
as we haven't done anything yet!

Once you're in a directory, it's easy to move back by remembering
that, in BASH, \texttt{.} refers to your current directory, and
\texttt{..} refers to one directory back from the present
one. \texttt{.} will come in handy later, but for now just move back
one directory (\texttt{cd ..}) and list the contents of this
directory. If you ever got lost, using \texttt{cd} with no arguments
(\emph{i.e.}, don't give it a directory to move into; \texttt{cd})
will take you back to what is called your \emph{home} directory. Move
back home now. \texttt{\textasciitilde} is a useful shortcut for your home
directory; \texttt{cd \textasciitilde/my.new.directory} will probably take you to
your new directory.
\subsection{File manipulation}
Creating files is easy. It's simple to create an empty file
(\texttt{touch name.of.file}), but often you will want to do more than
that. \texttt{nano} is a lightweight text editor; run \texttt{nano
  test.txt} to open it in a new file. Write some nonsense, then exit
by pressing \emph{control-x} (the commands are at the bottom of the
file). Follow the prompts; \texttt{y}es you do want to save, you do
want to save in the file you opened (hit enter), and you're back in
BASH. You can now print the contents of the file (\texttt{cat
  test.txt}) by con\emph{cat}enating files and writing their output
(more in a moment). You can even quickly scroll through the contents
of a file (\texttt{less test.txt}) and use \texttt{q} to \emph{quit},
but this is only useful if we want to quickly examine a large file.

\emph{Copy} your file (\texttt{cp test.txt new.copy.txt}); verify
using the above information that the file is the same. We can also
\emph{move} a file to a new location (\texttt{mv new.copy.txt
  \textasciitilde/my.new.directory/}), and even give it a new name
(\texttt{mv new.copy.txt new.name.txt}); note that moving a file so it
has a new name is the same as renaming it! Concatenate both of those
files together now (\texttt{cat one.file another.file}). You can also
\emph{remove} (delete) a file (\texttt{new.name.txt}); there is no
undo in BASH, and there are no confirmation prompts before
deleting. To remove a directory, you must first make it empty and then
use \texttt{rmdir}. There is much more to learn on this topic, and I
have put some new information in the appendix. File manipulation
commands are very powerful: do not run before you can walk, and don't
start trying to move hundreds of files around at once, or delete files
according to what they contain (all things you can do easily!) until
you are comfortable with the above. \emph{There is no undo!}
\subsection{Running a program and permissions}
One of the reasons UNIX-like computers run the Internet is they have a
very strong \emph{permissions} system. Permissions what user can read,
write, and run (start a program) a file.

Create a new file called \texttt{silly.sh} (\texttt{sh} stands for
\emph{shell} script), and fill it with the following text:

\begin{lstlisting}
  #!/usr/bin/env bash
  echo 'hello, world'
\end{lstlisting}

The first line (called the \emph{shebang}, as in ``the whole
shebang'') tells BASH that this is a BASH script; please just treat it
as a magic incantation whose meaning doesn't matter. The next line is
BASH code; see what happens when you type the words into your console
now and its meanning will become apparent.

Right now, BASH won't let us run our program, but try anyway by typing
\texttt{./silly.sh} (translated as: please run \texttt{./}---this
directory, \texttt{silly.sh}---this file). List the contents of your
directory with more detail (\texttt{ls -l}) and you'll see a column
with all the files, the dates when they were last editted, who
\emph{owns} them, and at the far left a series of confusing
\texttt{r}s (read), \texttt{w}s (write), and \texttt{x}s (execute/run)
permissions. They're repeated because different kinds of users have
different permissions: in general, you can do anything to a file you
create, but not to someone else's, and the administrator can do
anything to anyone.

Let's \emph{change} the file \emph{mode}---give ourselves permission
to \emph{execute} the file---by typing \texttt{chmod +x
  silly.sh}. That's it; when you \texttt{ls} the directory, you'll
probably see the colour of the file has changed (green for go!), and
now you can run our silly program.

This is the most difficult thing you'll have to do, and it probably
seems silly now, but it is important. Other users can't modify your
files; if you were to try and modify system settings, or other users'
files, you would get an error. This also helps keep the system safe;
before modifying anything sensitive, adminstrators have to
authenticate (type type \texttt{sudo}, for \emph{super user do}) to do
something dangerous. This way all your files and settings are safe, so
long as the administrator knows what they're doing.
\section{Logging in (SSH)}
SSH (\emph{secure shell}) is a safe way to log into a remote
computer---it's a way of opening a shell (see above) on that
computer. You can think of an SSH connection as a quantum
\emph{tunnel}; once you open that tunnel, it's like you're sat at the
computer you're tunnelled into. If you want to use a file on your own
computer, you must first copy it there, because you're no longer on
your own computer, you're on the remote computer.

From a UNIX-like computer (MacOS, Linux, etc.), openning a connection
is as simple as typing \texttt{ssh username@remote.computer}. On
Windows, you will have to download a program like \emph{PuTTy}; the
warnings about the program being illegal reflect US foreign policy,
and if this concerns you I encourage you to Wikipedia the program. So,
to tunnel into my computer, I would type \texttt{ssh will@Lance}, or
if I only knew the IP address of the remote computer (not its name) I
would use that (\texttt{ssh will@123.345.678.12}). To copy a file, I
would use \emph{secure copy} which is based around SSH; something like
\texttt{scp file.to.copy user@computer:/destination/path/}. Note that
I have to say where I want to send the file on the remote computer!
Alternatively, I could connect to the remote computer over FTP (either
through the command line or using a program like CyberDuck). When
typing your password into SSH, you often won't se any asterisks or
anything else appearing on screen; this is normal. To log out, simply
\texttt{logout}.

Using passwords is very risky, because they're quite easy to
crack. Many administrators would prefer you use SSH keys; simply-put,
you have a \emph{private key} that can be used to generate a
\emph{public key}. There's some magic, complicated maths going on, but
with a public key anyone can check that you own the private key, yet
it's very hard for someone to figure out what the private key looks
like using the public key. So keys are a good way to check someone is
who they say they are. Don't think about how this works, just do the
following:

\begin{itemize}
\item Get an SSH key.
\item \emph{ Never share your private key with anyone ever}. If you
  do, tell everyone who uses that key to no longer trust it.
\item Find your remote computer's administrator and ask them to use
  your public key for login.
\end{itemize}

Doing this means you never need to type your password when you log in
through SSH, and makes everything much safer. People are always trying
to hack big, fancy computers for their own nefarious purposes, and
this will help keep everyone safe.
\section{Running programs in parallel}
You can run a program in the background by putting an ampersand after
the line (\emph{e.g.}, \texttt{echo 'hello, world' \&}). This means
you can carry on doing other pieces of work, or you can \emph{run
  multiple programs at the same time}. Make sure you don't run more
programs than you computer has processors!

You can also set a program to not stop running, even when you log out,
by using \texttt{nohup} (\emph{e.g.}, \texttt{nohup echo 'hello,
  world' \&}). Your program's output will be put into a file called
\texttt{nohup.out}. Be a \emph{nice person} and use \texttt{top} or
\texttt{htop} to check the computer's load before doing this too much,
though.

If you're using \emph{R}, make sure you use \texttt{mclapply} from
\emph{parallel} to run things. Not only does it make your code easier
to read, but it also means you're using all the cores on your
computer. That will make things faster; otherwise, what was the point
in your learning BASH to begin with?
\section{Useful BASH commands and tricks}
\begin{tabular}{l p{14cm}}\hline
  \textbf{Command} & \textbf{Description}\\\hline
  \texttt{top} & Lists all the processes (programs) running on the computer, and how much of the processor is being used. \texttt{htop} is a much more friendly graphical version of \texttt{top}; if it's installed on your system use it. You can also stop programs here, a bit like control-alt-delete on a windows computer.\\
  \texttt{make} & Used to compile a lot of programs; typically just typing \texttt{make} in the directory that contains the source code of a C/C++ or Fortan program will be sufficient. The name comes from the fact it looks for a \emph{makefile} that describes how to compile the program you want to use. Try \texttt{make -f name.of.makefile} if that doesn't work, there's often more than one Makefile and you need to pick which version of the program you want to compile.\\
  \texttt{wget} & \emph{Get} something from the \emph{Web}; use it to download programs etc.\\
  \texttt{tar} & Used to create `tar-balls' (zip files); \texttt{tar -xf file.tar.gz} will unzip something, \texttt{tar -cf new.tar folder another.file} will create a new .tar from the folder(s) and file(s) you specify. \texttt{unzip} does the same thing but for zip file.\\
  \texttt{rm} & You can delete a whole folder and everything in it by running \texttt{rm} \emph{recursively} and \emph{forcing} it to delete things, \emph{e.g.}, \texttt{rm -rf folder}. Many, many people have deleted more than they bargained using this tool.\\
  \texttt{cp} & You can \emph{recursively} copy a directory, \emph{e.g.}, \texttt{cp -r folder}.\\
  \texttt{Rscript} & Runs an R script; useful in conjunction with \texttt{nohup} and friends (see above).\\
    \texttt{apt-get} & Used on many Linux computers (notably Debain and Ubuntu) to install programs; you will almost certainly need to have \texttt{sudo} access to do this.\\
  \texttt{sudo} & \emph{Super user do}; allows administrators to authenticate before doing anything potentially dangerous to the system. If it's your own computer, you will be able to use this command. For the love of God be careful with it!\\
  \texttt{tab} & Hit the tab-key when part-way through a command or file-name and BASH will try to complete it for you. If there is more than one potential match, keep hitting it and it will show all the possible completions.\\
  \texttt{*} & You can list, copy, and delete based on wild-card matching. For example, \texttt{ls *.txt} will show you all the text files in a directory, and \texttt{rm first.try*} will remove files starting with the phrase `first try'. Many, many people have done more than they bargained using this tool.\\
  \texttt{|} & Pipes are one of the most powerful features of BASH. You can use them to chain commands together; for instance \texttt{ls -l | wc -l} will print out all the files in your directory (one per line; \texttt{ls -l}), and \texttt{|} takes that information and passes it to \texttt{wc -l} that counts the number of lines in its input. Voil\'{a}; you now know how many files are in the directory. Doing operations using \texttt{|} can result in orders of magnitude increase in execution speed; writing to disk is slow.\\
  \texttt{who}/\texttt{write} & To send a message to someone logged in, find them with \texttt{who} and note their terminal (second column). Send them a message with \texttt{write username terminal}, hit enter, type your message, then control-D (end-of-line).\\
\texttt{R env} & Set the default \emph{R} library by making a file called \texttt{.Renviron} containing the line \texttt{R\_LIBS=/home/share/R/} (...or wherever else...)\\
  \hline
  \end{tabular}

\end{document}
%%% Local Variables:
%%% mode: latex
%%% TeX-master: t
%%% End:
